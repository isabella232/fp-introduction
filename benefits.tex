\begin{frame}
\begin{block}{Why?}
\begin{center}
Why would we do this? What are the practical benefits?
\end{center}
\end{block}
\end{frame}

\begin{frame}
\begin{block}{Why FP?}
\begin{center}
\begin{itemize}
\item<1-> the practical benefits are not always immediately obvious
\item<2-> this is especially true when given trivial examples, such as summing a list
\item<3-> but is there a point to all this?
\item<4-> a benefit to throwing away familiar tools, and replacing them?
\end{itemize}
\end{center}
\end{block}
\end{frame}

\begin{frame}
\begin{center}
\ldots
\end{center}
\end{frame}

\begin{frame}
\begin{block}{Some general benefits are}
\begin{center}
\begin{itemize}
\item<1-> an ability to \emph{reason} about \emph{discrete} programs and sub-programs i.e. \emph{local reasoning}
\item<2-> an ability to \emph{compose} sub-programs to make slightly less small programs, \emph{indefinitely}
\end{itemize}
\end{center}
\end{block}
\end{frame}

\begin{frame}
\begin{block}{What are the benefits of FP?}
\begin{center}
Although this question commands a considerable amount of work, it is a seemingly endless rabbit hole, for which I have never found the bottom \ldots
\end{center}
\end{block}
\end{frame}
